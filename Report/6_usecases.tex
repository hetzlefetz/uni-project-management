\chapter{User Stories}

Durch Brainstorming und Auswertung der Interviews haben wir folgende User Stories entwickelt

\section{Userstorymap}

\section{Glossar für die User Stories}
\begin{itemize}
\item \textbf{[MEETING]} : Eine Instanz eines Weekly Stand Ups
\item \textbf{[USER]} : Teilnehmer eines \textbf{[MEETING]}s 
\item \textbf{[PUNKT]} : Eine der drei möglichen Kategorien die Abgefragt werden (Was habe ich getan? Was werde ich tun? Was blockiert mich?)
\end{itemize}
\section{User Story Nr.1} 


Als \textbf{[USER]} möchte ich die Möglichkeit haben mit Mentions user oder meetings zu referenzieren

\section{User Story Nr.2}


Als \textbf{[USER]} möchte ich ein \textbf{[PUNKT]} markieren können (z.b.: ja, nein, lachen, traurig)
\section{User Story Nr.3}


Als \textbf{[USER]} möchte im \textbf{[MEETING]} die Teilnehmer und Datumsangaben sehen
\section{User Story Nr.4 }


Als \textbf{[USER]} möchte ich ein \textbf{[PUNKT]} bearbeiten oder löschen im aktuellen Meeting
\section{User Story Nr.5}


Als \textbf{[USER]} möchte ein Eingabefeld haben mit (Emojisupport) um meine \textbf{[PUNKT]}e einzutragen
\section{User Story Nr.6}


Als \textbf{[USER]} möchte ich einen \textbf{[PUNKT]} für das nächste \textbf{[MEETING]} eintragen
\section{User Story Nr.7}


Als \textbf{[USER]} möchte ich ein \textbf{[MEETING]} öffnen können um mehr Details erhalten
\section{User Story Nr.8}


als \textbf{[USER]} möchte ich vergangene \textbf{[MEETING]}s durchsuchen können (Schlagwortsuche)
\section{User Story Nr.9}


Als  \textbf{[USER]} möchte ich \textbf{[MEETING]}s starten können welche \textbf{[PUNKT]}e enthalten
\section{User Story Nr.10}


Als  \textbf{[USER]} möchte ich einfach durch die Timeline navigieren wo nur das aktive \textbf{[MEETING]} expanded ist
\section{User Story Nr.11}


Als \textbf{[USER]} möchte ich eine Liste von \textbf{[MEETING]} in einer Timeline sehen

\section{User Story Nr.12}

Als \textbf{[USER]} möchte ich ein  \textbf{[PUNKT]} als "wichtig" markieren