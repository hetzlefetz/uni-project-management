\chapter{Einführung}

Zielsetzung unseres Teams (DEV-2) war gemeinsam ein Feature umfänglich für das \ac{SPMS} zu entwickeln. Nach einem initialen Brainstorming haben wir uns für die Entwicklung eines Weekly Standup Moduls entschieden. Hierzu haben wir zuerst eine Marktanalyse durchgeführt, um zu eruieren, welche Features bereits am Markt vorhanden sind. In einem nächsten Schritt haben wir uns überlegt, wer unser Zielnutzer sein werden und haben auf dieser Grundlage Personas erstellt. Davon ausgehend haben wir Interviews mit potenziellen Nutzern geführt, um ein besseres Verständnis für diese zu entwickeln. Basierend auf unseren Ideen, der Marktanalyse, den Personas und den Interviews haben wir dann die User Stories entwickelt und zur Ausarbeitung unter uns im Team aufgeteilt.\\
Wir haben uns für einen agilen  Entwicklungsprozess entschieden und in kurzen Sprints an den Features zu arbeiten. Hierzu haben wir uns meist vor oder nach den wöchentlichen Vorlesungen zusammen gesetzt oder wenn einmal ein größeres Feature entwickelt werden musste haben wir uns am Wochenende getroffen.\\
Unsere Zusammenarbeit haben wir meist remote mittels Discord\cite{discord2022} koordiniert und auf einem Miroboard\cite{miro2022} durchgeführt. 