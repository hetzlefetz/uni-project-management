\chapter{Interviews}

Um Bedürfnisse realer Benutzer von vorne herein zu Berücksichtigen haben wir mit Usern ein, durch ein Fragebogen begleitetes Interview durchgeführt.

\section{Fragebogen und Ergebnisse}

Folgende Informationen haben wir mittels eines Fragebogens abgefragt.

\subsection{Frage 1: Benutzt du Weekly Standups?}
\subsection{Frage 2: Welcher Nutzergruppe gehörst du an?}
\subsection{Frage 3: Welche Projektarten führst du aus?}
\subsection{Frage 4: Welche Projektarten führst du aus?}
\subsection{Frage 5: Wie alt bist du?}      
\subsection{Frage 6: Wie viel Zeit verbringst du Wöchentlich in Standup Meetings?}
\subsection{Frage 7: Wie hilfreich findest du Weekly Standups?}
\subsection{Frage 8: Wie arbeitest du zur Zeit?}
\subsection{Frage 9: Wie bewertest du folgende Features?}
\subsubsection{Frage 9.1: Erinnerung}
\subsubsection{Frage 9.2: Timeline}
\subsubsection{Frage 9.3: Suche}
\subsubsection{Frage 9.4: Markieren}
\subsubsection{Frage 9.4: Vorgegebene Eingabefelder}
\subsection{Frage 10: Was ist dein bevorzugter Zeitpunkt für ein Weekly Standup?}
\subsection{Frage 11: Welche Form eines Weekly Standups würdest du bevorzugen?}
\subsection{Frage 12: Würdest du einen Chatbot als hilfreich empfinden, welcher dir Informationen wie Deployment, Status, Logereignisse im Meeting mitteilt?}
\subsection{Frage 13: Welche Informationen wären für dich als Projektmanager in einem Weekly Standup wichtig?}
\subsection{Frage 14: Wenn du bereits Weekly Standups gemacht hast, welche Tools hast du dafür benutzt?}
\subsection{Frage 15: Wie wichtig wäre dir eine Zeitbegrenzung des Meetings?}

\section{Methodik}



